\documentclass[10pt,journal,compsoc]{IEEEtran}
\usepackage{amsmath,amsfonts}
\usepackage{algorithmic}
\usepackage{array}
\usepackage[caption=false,font=normalsize,labelfont=sf,textfont=sf]{subfig}
\usepackage{textcomp}
\usepackage{stfloats}
\usepackage{url}
\usepackage{verbatim}
\usepackage{graphicx}
\usepackage[backend=biber,style=ieee]{biblatex}
\addbibresource{bibtex/bib/references.bib}
\hyphenation{op-tical net-works semi-conduc-tor IEEE-Xplore}
\def\BibTeX{{\rm B\kern-.05em{\sc i\kern-.025em b}\kern-.08em
    T\kern-.1667em\lower.7ex\hbox{E}\kern-.125emX}}
\usepackage{balance}
\begin{document}
\title{Detección de TDAH utilizando IA}

\author{María~Fernanda~Barriba~Vega,
    Miguel~Angel~Cuevas~González,
    y~Marco~Antonio~León~Rodríguez% <-this % stops a space
    \thanks{This work is licensed under CC BY 4.0. To view a copy of this license, visit https://creativecommons.org/licenses/by/4.0/}%
    \thanks{\indent M.F. Barriba Vega, M.A. Cuevas González, y M.A. León Rodríguez están con la Facultad de Ciencias Químicas e Ingeniería, Universidad Autónoma de Baja California, México (e-mail: barribam@uabc.edu.mx; mcuevas80@uabc.edu.mx; marco.leon11@uabc.edu.mx).}}

\markboth{Tópicos Selectos de Investigación, Protocolo de Investigación, \today}%
{Barriba \MakeLowercase{\textit{et al.}}: Protocolo de Investigación}

\maketitle

\begin{abstract}
    This document describes the most common article elements and how to use the IEEEtran class with \LaTeX \ to produce files that are suitable for submission to the Institute of Electrical and Electronics Engineers (IEEE).  IEEEtran can produce conference, journal and technical note (correspondence) papers with a suitable choice of class options.
\end{abstract}

\begin{IEEEkeywords}
    Machine learning, artificial intelligence, ADHD, diagnosis, intervention.
\end{IEEEkeywords}

\section{Introducción}
\IEEEPARstart{A}{quí} va la introducción. Ya que tengamos la información necesaria, la escribiremos.

\section{Justificación}
El Trastorno por Déficit de Atención e Hiperactividad (TDAH), representa una preocupación significativa en toda la población, ya que afecta el rendimiento académico y la calidad de vida de quienes lo padecen \cite{aparicio2024tecnologias}. Es uno de los trastornos del neurodesarrollo con más alta prevalencia, estimada en 5\%, en la población infantil \cite{boechi2023tecnologias}. El TDAH tiene un impacto significativo en la vida de las personas, incluyendo dificultades en la escuela, en el trabajo y en las relaciones interpersonales \cite{rivera2016elevada}. Por lo tanto, es importante detectar y tratar el TDAH de manera efectiva desde niños. Este trastorno se caracteriza por dificultades en la atención, la concentración y el control de impulsos; estos síntomas llevan a problemas en los entornos ya mencionados \cite{santarrosaanalisis}. Esta discapacidad no tiene cura, pero puede ser gestionada eficazmente, y algunos síntomas pueden mejorar a medida que las personas crecen, según el National Center on Birth Defects and Developmental Disabilities (NCBDDD) en 2023.

En cuanto a los métodos de diagnóstico y tratamiento clásico, la OMS recomiendan que las intervenciones de primera línea para el TDAH sean las no farmacológicas \cite{normasfarmacologicas} (ambientales, sociales, conductuales y psicológicas) y que solo se sugiere la derivación a un especialista para considerar la prescripción del metilfenidato en caso de que no sean eficaces las intervenciones no farmacológicas \cite{boechi2023tecnologias}.

Adelman y Vogel (citados por González et al., 2010, p.325) señalan que entre el 40 \% y el 56 \% de los estudiantes de la ESO (Educación Secundaria Obligatoria), institución en españa \cite{sistemaeducativoespana}, que padecen una o varias DEA (Dificultades Específicas de Aprendizaje) abandonan los estudios, frente a un 25 \% de abandono académico en estudiantes sin dificultades \cite{consecuenciassociales}. Estos datos hacen necesaria la investigación sobre este tema, ya que estas dificultades y el fracaso o abandono escolar que implican, pueden llegar a afectar a toda la trayectoria vital de estos jóvenes, tanto a nivel laboral, como económico, psicológico, emocional y social.

\section{Planteamiento}
Este trabajo tiene como objetivo examinar en profundidad el estado actual de la investigación en el ámbito del Trastorno por Déficit de Atención e Hiperactividad (TDAH). Para lograr esto, se realizará una revisión exhaustiva de los estudios existentes que analicen el uso de técnicas avanzadas basadas en aprendizaje automático (machine learning) con el propósito de mejorar la precisión en la detección y diagnóstico del TDAH. La investigación se centrará en evaluar cómo estas técnicas emergentes pueden superar las limitaciones de los métodos tradicionales.

Además de la revisión de técnicas de detección, se investigarán y analizarán diversas estrategias de intervención personalizadas propuestas en estos estudios. El objetivo es comprender cómo estas intervenciones pueden ser implementadas para brindar un apoyo más eficaz a las personas afectadas por el TDAH, abordando sus necesidades específicas y mejorando su calidad de vida.

El trabajo tiene como objetivo proponer -partiendo del marco teórico-, una solución efectiva para el diagnóstico de TDAH, que involucre el uso de inteligencia artificial. Esta propuesta servirá para que otros investigadores y personas interesadas en el tema (que cuenten con los recursos necesarios) puedan implementarla con éxito, teniendo con certeza que será efectiva en la práctica. Además de la propuesta para el diagnóstico de TDAH, se prevé encontrar técnicas/métodos que ayuden a fomentar el uso de la tecnología encontrada en la práctica.

El problema principal que se abordará es la dificultad para realizar un diagnóstico temprano y preciso del TDAH, dado que los métodos tradicionales suelen depender de evaluaciones subjetivas y pueden retrasar el tratamiento adecuado \cite{khullar2021deep}. Este trabajo tiene como meta proponer una solución efectiva que supere estas barreras y ofrezca un enfoque más fiable y eficiente para el diagnóstico y la gestión del TDAH.

\section{Marco Teórico}
\subsection{Contexto histórico}
El TDAH (Trastorno por Déficit de Atención e Hiperactividad) es un trastorno caracterizado por una falta de atención, hiperactividad e impulsividad que dificulta el desarrollo del individuo \cite{thapar2008overview}. A su vez, va acompañado de otros trastornos psiquiátricos, particularmente de conducta o del desarrollo, como el autismo, depresión, ansiedad, entre otros. Comienza en la infancia, pero se ha comprobado que suele persistir en la adolescencia y la vida adulta \cite{combs2015perceived}.

Usualmente es complicado diagnosticar a un paciente a temprana edad debido a que los síntomas son comportamientos que para una persona que se encuentra en la etapa de primera infancia se consideran normales \cite{national-institute-of-mental-health-no-date}. Como su nombre lo menciona, esta discapacidad se caracteriza por dos síntomas principales que definen sus dos tipos: la falta de atención, y la hiperactividad-impulsividad \cite{combs2015perceived}. A pesar de que la mayoría de personas experimentan algún síntoma similar en algún momento, el TDAH se diagnostica cuando los síntomas son suficientemente notorios, continuos y sobre todo graves \cite{chen-2023}. Independientemente del tipo de TDAH, los síntomas se asocian generalmente a un deterioro persistente en varios ámbitos de la vida del individuo, lo que suele provocar un estrés elevado. En concreto, el estrés surge cuando una persona se agobia por la percepción que tiene sobre las exigencias del entorno, superando sus recursos, por lo tanto se le define como un estado cognitivo emocional negativo que dificulta al individuo a gestionar los acontecimientos que suceden en la vida cotidiana \cite{combs2015perceived}.

A lo largo de las investigaciones sobre el diagnóstico del TDAH se han desarrollado criterios para la realización del diagnóstico del TDAH específicamente en adultos \cite{combs2015perceived}, considerando en gran medida los déficits del funcionamiento ejecutivo del individuo y argumentando que estos identifican el TDAH en la edad adulta mejor que los criterios del Manual Diagnóstico y Estadístico de los Trastornos Mentales \cite{american2000diagnostic}. La ICD-10 (International Classification of Diseases, Tenth Revision) propone tres conjuntos de síntomas: falta de atención, hiperactividad e impulsividad, los cuales deben estar presentes para el diagnóstico de TDAH y deben ser exhibidos en más de un entorno, como el hogar o la escuela. Por otro lado, el DSM-5 (Diagnostic and Statistical Manual of Mental Disorders, Fifth Edition) distingue tres presentaciones diferentes del TDAH, predominantemente inatento, predominantemente hiperactivo-impulsivo, o combinado. No es necesario que los tres conjuntos de síntomas estén presentes al mismo tiempo \cite{doernberg2016neurodevelopmental}.

\subsection{Metodologías}
Actualmente, se utilizan cuestionarios específicos para adultos, como el Conners Adult ADHD Rating Scales (CAARS) o el Vanderbilt Assessment Scale (VAS), para evaluar los síntomas de atención, hiperactividad e impulsividad \cite{review2003neuropsychology}. El cuestionario CAARS aplica únicamente para personas con 18 años o más, esto para que sean considerados adultos. Debido a la antigüedad del cuestionario, y sus escasas actualizaciones, este será descontinuado a partir de Octubre 31 de 2024 \cite{scales2024adult}. El cuestionario VAS se aplica únicamente en niños y adolescentes, tomando en cuenta lo establecido por el Manual Diagnóstico y Estadístico de los Trastornos Mentales, Cuarta Edición (DSM-IV), distinguida por su enfoque integral, incorporando aportes tanto de los padres como de los docentes para proporcionar una visión multifacética del comportamiento del niño en diferentes entornos \cite{vanderbilt2024tools}. Ambos cuestionarios tienen algo en común, el cual es que ambos son aplicados directamente hacia la persona objetivo, y necesitan de un tercero para llevarla a cabo y anotar las respuestas. Las mencionadas son una pequeña cantidad de la amplia diversidad de evaluaciones y cuestionarios. Existen pruebas neuropsicológicas, como el Test de Stroop: también conocido como Prueba de Colores y Palabras, fue desarrollado por Golden con el objetivo de evaluar elementos como la atención selectiva y el control inhibitorio \cite{mente2024conners}.
Consiste en mostrar palabras de colores (como "rojo", "azul") que están impresas en un color diferente al que describen (por ejemplo, la palabra "rojo" escrita en azul). La persona debe decir el color de la tinta y no leer la palabra, lo cual es más difícil porque el cerebro automáticamente tiende a leer la palabra en lugar de centrarse en el color; mide la capacidad de controlar la atención y manejar la interferencia \cite{test2024stroop}.
Dichos cuestionarios pueden tomar un tiempo significativo para tanto la recolección de datos, y al mismo tiempo el análisis y la disposición de una respuesta final. Es por eso que las nuevas herramientas con inteligencia artificial y análisis de datos podrían hacer la diferencia en destruir dicha brecha.

\subsection{Antecedentes}
La aplicación de modelos de aprendizaje de máquina para el diagnóstico de ADHD es algo muy reciente en el campo; normalmente se utilizan un sistema basado en conocimiento con reglas si-no (por ejemplo), los cuales permiten tener una mejor trazabilidad hacia él, otorgando una explicación más interpretable tanto para el paciente como para el médico en cuestión sobre el resultado.

De acuerdo a un estudio clínico realizado en el Reino Unido \cite{chen-2023}, el modelo más preciso de diagnóstico se consiguió utilizando un método híbrido, que funciona a partir de un modelo de aprendizaje de máquina en conjunto con un sistema basado en conocimiento.

Otro estudio publicado en el 2012 \cite{wall-2012}, menciona conseguir entrenar un modelo de aprendizaje de máquina para el diagnóstico de TDAH con una precisión de casi 100\%, pero queda en cuestión la practicidad de estos resultados ya que utilizaron un total 991 datos de pacientes diferentes -de los cuales 75 no cumplen con los criterios para un diagnóstico de TDAH-, complicando la medición real del comportamiento del modelo con datos fuera de esta población.

Para recomendar qué técnica o implementación de modelo de aprendizaje máquina utilizar, se tomará en cuenta la metodología de ciencia de datos Cross-Industry Standard Process for Data Mining (CRISP-DM), ya que es el estándar predeterminado que produce los mejores resultados para este tipo de proyectos \cite{schroer-2021}. Lo inconveniente es que la mayoría de estudios en el área no incluye la fase de lanzamiento (o deployment), pero este tiene sentido: para el contexto de investigación no es relevante como se lanza a producción el modelo, lo relevante son los descubrimientos y la metodología que llevaron a la conclusión de utilizar ese modelo.

\section{Hipótesis}
El uso de técnicas avanzadas de deep learning permitirá mejorar significativamente la precisión en la detección de TDAH en adultos y niños. A través de un análisis exhaustivo de la literatura y pruebas comparativas, se identificará un enfoque que optimice la generalización del modelo y reduzca el sobreajuste, mejorando la capacidad de diagnóstico en diferentes grupos etarios.

\section{Objetivos}
\subsection{Objetivo General}
Comparar el rendimiento de los algoritmos de IA en términos de precisión en los diagnósticos de TDAH a través de un análisis literario, evaluando su efectividad en comparación con técnicas comúnmente utilizadas en el área médica y analizando su aplicación en distintos grupos etarios

\subsection{Objetivos Específicos}
\begin{enumerate}
    \item Realizar una comparación exhaustiva del estado del arte que contraste la precisión y generalización de diferentes modelos, analizando los resultados en función de características específicas de cada grupo.
    \item Investigar las técnicas de deep learning utilizadas en la actualidad en la detección del TDAH en adultos y niños, identificando sus limitaciones en términos de precisión, eficiencia y generalización en ambos grupos.
    \item Proponer mejoras o ajustes en los modelos analizados, con base en los resultados obtenidos para optimizar su precisión y reducir el margen de error en diagnósticos de TDAH.
\end{enumerate}

\section{Metas de Investigación}

\subsection{Meta General}
Desarrollar una propuesta basada en inteligencia artificial que mejore la detección y diagnóstico del TDAH, superando las limitaciones de los métodos tradicionales mediante un análisis de diferentes técnicas de aprendizaje automático.

\subsection{Metas Específicas}
\begin{enumerate}
    \item Revisión exhaustiva: Realizar un análisis exhaustivo de la literatura actual sobre la aplicación de técnicas de IA en la detección del TDAH, identificando sus fortalezas y debilidades.
    \item Comparación de Modelos: Comparar el rendimiento de diversos algoritmos de aprendizaje automático en términos de precisión y generalización para el diagnóstico de TDAH en diferentes grupos de edades(niños y adultos).
    \item Propuestas de Mejora: Sugerir ajustes o nuevas metodologías, como la implementación de optimizadores avanzados (por ejemplo, AdamW), que optimicen la precisión y eficiencia de los modelos de diagnóstico.
    \item Marco Teórico para la Implementación: Proporcionar un marco teórico y metodológico que sirva como base para futuros estudios y aplicaciones en el diagnóstico del TDAH.
\end{enumerate}

\section{Metodología}

\subsection{Diseño de la Investigación}
No experimental, transaccional y exploratoria. La investigación se centrará en el análisis de estudios existentes y la comparación de resultados.

\subsection{Fases de la Metodología}
\begin{enumerate}
    \item Revisión de la literatura
          \begin{enumerate}
              \item Recopilar y analizar artículos académicos, estudios de caso y reportes sobre el uso de IA en el diagnóstico de TDAH.
              \item Identificar las técnicas y modelos de aprendizaje automático más utilizados y sus resultados.
          \end{enumerate}
    \item Análisis Comparativo
          \begin{enumerate}
              \item Establecer criterios de comparación (precisión, generalización, eficiencia) entre los diferentes modelos analizados.
              \item Clasificar los estudios por tipo de modelo, grupo etario y método de evaluación utilizado.
          \end{enumerate}
    \item Identificación de Mejores Prácticas
          \begin{enumerate}
              \item A partir de los hallazgos de la revisión, identificar las mejores prácticas en el uso de IA para el diagnóstico del TDAH.
              \item Evaluar cómo técnicas como deep learning y optimizadores avanzados pueden mejorar la detección.
          \end{enumerate}
    \item Desarrollo de la Propuesta
          \begin{enumerate}
              \item Elaborar una propuesta que incluya recomendaciones sobre los modelos y técnicas a utilizar, así como sugerencias para su futura implementación.
              \item Incluir un marco teórico que justifique la elección de cada técnica y su potencial impacto en la detección del TDAH.
          \end{enumerate}
    \item Presentación de Resultados
          \begin{enumerate}
              \item Redactar un informe final que incluya los hallazgos, las comparaciones de los modelos, y la propuesta elaborada.
              \item Presentar la investigación de manera que sea accesible y útil para otros investigadores y profesionales en el área.
          \end{enumerate}
\end{enumerate}

\printbibliography

\begin{IEEEbiography}{María Barriba}
    María Barriba is currently pursuing her Bachelor's degree in Electrical Engineering at the Universidad Autónoma de Baja California, México. Her research interests include renewable energy systems, power electronics, and smart grid technologies.
\end{IEEEbiography}

\begin{IEEEbiography}{Miguel Cuevas}
    Miguel Cuevas is a student at the Universidad Autónoma de Baja California, México, where he is working towards his Bachelor's degree in Computer Science. His areas of interest include machine learning, artificial intelligence, and software development.
\end{IEEEbiography}

\begin{IEEEbiography}{Marco León}
    Marco León is an undergraduate student in the Department of Mechanical Engineering at the Universidad Autónoma de Baja California, México. His research focuses on robotics, automation, and control systems.
\end{IEEEbiography}


\end{document}
